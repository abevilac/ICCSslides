\documentclass{beamer}
\usepackage[utf8]{inputenc}

\def\dt{\partial t}
\def\dx{\partial x}
\def\dy{\partial y}

\def\dz{\partial z}
\def\BE{\begin{equation}}
\def\EE{\end{equation}}
\def\half{\frac{1}{2}}
\def\calT{\cal T}
\def\Deltax{\Delta x}
\def\Deltat{\Delta t}
\DeclareSymbolFont{largesymbolsA}{U}{txexa}{m}{n}
\DeclareMathSymbol{\varprod}{\mathop}{largesymbolsA}{16}
\setbeamerfont{caption}{size=\scriptsize}

\renewcommand{\thefootnote}{*}

\title{\bf{Analyzing Complex Models using Data and Statistics}\thanks{ICCS 2018, LNCS 10861, pp. 724–736, 2018.}}

\author{Abani Patra\textsuperscript{1,3}, Andrea Bevilacqua\textsuperscript{2}, Ali Akhavan Safaei\textsuperscript{1}}
\institute{\textsuperscript{1} Department of Mechanical and  Aerospace Engineering\\ \textsuperscript{2} Department of Earth Sciences\\ \textsuperscript{3}Institute for Computational Data Sciences\\\vskip.1cm University at Buffalo, NY 14260, USA}

\date[ICCS 2018]{International Conference on Computational Sciences 2018\vskip.2cm \small{11-13 June 2018, Wuxi, China}}

\begin{document}

\frame{\titlepage}



\begin{frame}
\frametitle{Models and assumptions}
\textbf{What is a Model?}
\begin{quote}{\it A model is a representation of a postulated relationship among inputs and outputs of a system, usually informed by observation and based on a hypothesis that best explains the relationship.}\end{quote}
\vskip.5cm
\begin{itemize}
\item models depend on a {\it hypothesis}, and,
\item models use the {\it data from observation} to validate and refine the hypothesis.
\end{itemize}
\end{frame}


\begin{frame}
\frametitle{Analysis Process - \small{in \emph{predictive mode}}}
We are interested in the general predictive capabilities of the models, related to their outcomes over a whole range.
\vskip.3cm
\begin{itemize}
  \item Stage 1: Set parameter Ranges $P_M\left(p_1,\dots,p_{N_M}\right)\sim \bigotimes_{i=1}^{N_M} Unif(a_{i,M},b_{i,M}).$
  \vskip.5cm
  \item Stage 2: Run Simulations and Gather Data
\end{itemize}
\vskip-.2cm
\begin{figure}[H]
\includegraphics[width=0.85\textwidth]{figures/modelproc.png}
\caption{Models and variables}
\end{figure}
\begin{itemize}
  \item Stage 3: Analyze Results
\end{itemize}
\end{frame}


\begin{frame}
\frametitle{Statistics of latent variables - \small{dominance factors}}
Dominance factors provide insight into the largest latent variable, as a function of time, space, model and parameters.
\vskip.5cm
\begin{definition}[dominance factors]
Let $(F_i)_{i\in I}$ be random variables on $(\Omega, \mathcal F, P_M)$. Then, $\forall i$, the dominant variable is defined as:
$$\Phi:=\left\{
    \begin{array}{ll}
      \max_i |F_i|, & \hbox{if not null;} \\
      1, & \hbox{otherwise.}
    \end{array}
  \right.$$
In particular, for each $j \in I$, the dominance factors are defined as:
$$p_j:=P_M\left\{\Phi=|F_j|\right\}.$$
\end{definition}
\end{frame}

\begin{frame}
\frametitle{Statistics of latent variables - \small{expected contributions}}
Random contributions are obtained dividing the latent variables by the dominant variable $\Phi$, and hence belong to $[0,1]$.
\vskip.5cm
\begin{definition}[expected contributions]
Let $(F_i)_{i\in I}$ be random variables on $(\Omega, \mathcal F, P_M)$. Then, $\forall i$, the random contribution is defined as:
$$C_i:=\frac{F_i}{\Phi},$$
where $\Phi$ is the dominant variable. Thus, $\forall i$, the expected contributions are defined by $\mathbb E^{P_M}\left[C_i\right]$.
\end{definition}
\end{frame}


\begin{frame}
\frametitle{Modeling of geophysical mass flows}
The depth-averaged Saint-Venant equations are:
\small{
\begin{eqnarray}
\label{eq:D_A}
\frac{\partial h}{\partial t} +
\frac{\partial}{\partial x}(h \bar{u}) +
\frac{\partial}{\partial y}(h\bar{v}) &=& 0 \nonumber \\
\frac{\partial}{\partial t} (h\bar{u}) +
\frac{\partial}{\partial x}\left(h\bar{u}^2 + \frac{1}{2}k g_{z}h^2\right) + \frac{\partial}{\partial y}(h\bar{u}\bar{v}) &=& S_{x}\\
\frac{\partial}{\partial t} (h\bar{v}) +
\frac{\partial}{\partial x}(h\bar{u}\bar{v}) +
\frac{\partial}{\partial y}\left(h\bar{v}^2 + \frac{1}{2}k g_{z}h^2\right) &=& S_{y} \nonumber
\end{eqnarray}}

Source terms $S_x$, $S_y$ characterize \emph{Mohr-Coulomb} (MC), \emph{Pouliquen-Forterre} (PF) and \emph{Voellmy-Salm} (VS) models.
\end{frame}


\begin{frame}
\frametitle{Main assumptions - \small{all the models include \textit{curvature effects}}.}
\textbf{Mohr-Coulomb}
\begin{itemize}
\item \textit{Basal Friction} based on a constant friction angle.
\item \textit{Internal Friction} based on material yield criterion.
\end{itemize}

\textbf{Pouliquen-Forterre}
\begin{itemize}
\item \textit{Basal Friction} is based on an interpolation of two different friction angles, based on the flow regime and depth.
\item Normal stress is modified by a \textit{hydrostatic pressure force} related to the flow height gradient.
\end{itemize}

\textbf{Voellmy-Salm}
\begin{itemize}
\item \textit{Basal Friction} is based on a constant coefficient, similarly to the MC rheology.
\item Additional \textit{speed-dependent friction} is based on a quadratic expression.
\end{itemize}
\end{frame}


\begin{frame}
\frametitle{Overview of the case studies}
\begin{figure}
\includegraphics[width=1.05\textwidth]{ChineseFig.jpeg}
    \caption{{\bf [Left]} Inclined plane description, including local samples sites (red stars). {\bf [Right]}(a) Volc{\'a}n de Colima (M{\'e}xico) overview, with 51 numbered local sample sites (stars) and four labeled major ravines. Pile location is marked by a blue dot in both figures.}
\end{figure}
\end{frame}


\begin{frame}
\frametitle{Small scale flow - \small{observable outputs}}
\begin{figure}
\includegraphics[width=0.80\textwidth]{figures/incline/Height.png}
        \caption{Flow height in four locations. Bold line is mean value, dashed/dotted lines are 5$^{\mathrm{th}}$ and 95$^{\mathrm{th}}$ percentile bounds. Different models are displayed with different colors.}
\end{figure}
\end{frame}

\begin{frame}
\frametitle{Small scale flow - \small{power integrals}}
\begin{figure}
\includegraphics[width=0.95\textwidth]{figures/incline/PowersIncline.png}
        \caption{Spatial integral of the RHS powers. Bold line is mean value, dashed lines are 5$^{\mathrm{th}}$ and 95$^{\mathrm{th}}$ percentile bounds. Different models are displayed with different colors.}
\end{figure}
\end{frame}


\begin{frame}
\frametitle{Large scale flow - \small{proximal to the initial pile}}
\begin{figure}
        \includegraphics[width=0.80\textwidth]{figures/Colima/Pr1_total.png}
        \vskip-.3cm\caption{Dominance factors of \textbf{RHS} forces in three locations in the first km of runout. (a,d,g) assume MC; (b,e,h) assume PF; (c,f,i) assume VS. No-flow probability is displayed with a green dashed line.}
\end{figure}
\end{frame}

\begin{frame}
\frametitle{Large scale flow - \small{proximal to the initial pile}}
\begin{figure}
        \includegraphics[width=0.90\textwidth]{figures/Colima/Ci1_total.png}
        \small{\caption{Expected contributions of \textbf{RHS} forces in three locations in the first km of runout. (a,d,g) assume MC; (b,e,h) assume PF; (c,f,i) assume VS.}}
\end{figure}
\end{frame}

\begin{frame}
\frametitle{Large scale flow - \small{distal from the initial pile}}
\begin{figure}
        \includegraphics[width=0.80\textwidth]{figures/Colima/Pr2_total.png}
        \vskip-.3cm\caption{Dominance factors of \textbf{RHS} forces in three locations after 2 km of runout. (a,d,g) assume MC; (b,e,h) assume PF; (c,f,i) assume VS. No-flow probability is displayed with a green dashed line.}
\end{figure}
\end{frame}

\begin{frame}
\frametitle{Large scale flow - \small{distal from the initial pile}}
\begin{figure}
        \includegraphics[width=0.80\textwidth]{figures/Colima/Ci2_total.png}
        \vskip-.3cm\caption{Expected contributions of \textbf{RHS} forces in three locations after 2 km of runout. (a,d,g) assume MC; (b,e,h) assume PF; (c,f,i) assume VS.}
\end{figure}
\end{frame}

\begin{frame}
\frametitle{Large scale flow - \small{flow extent and spatial integrals}}
\begin{figure}
        \includegraphics[width=0.80\textwidth]{figures/Colima/AveragedColima.png}
        \caption{Spatial averages of $(a)$ flow speed, and $(b)$ Froude Number, in addition to the $(c)$ inundated area. Bold line is mean value, dashed/dotted lines are 5$^{\mathrm{th}}$ and 95$^{\mathrm{th}}$ percentile bounds. Different models are displayed with different colors.}
\end{figure}
\end{frame}


\begin{frame}
\frametitle{Conclusions}
\begin{itemize}
  \item we describe a \emph{prediction-oriented} approach, exploring a random family of simulations specified by the pair $\left(M, P_M\right)$.
\vskip.3cm
  \item our statistical framework processes the mean and the uncertainty range of either observable or latent variables in the simulation.
\vskip.3cm
  \item analysis is performed at selected sites, and spatial integrals were also performed, illustrating the characteristics of the entire output.
\vskip.3cm
  \item the new concepts of \emph{dominance factor} and \emph{expected contribution}, enable an informative description of the local dynamics.
\end{itemize}
\end{frame}

\end{document}
